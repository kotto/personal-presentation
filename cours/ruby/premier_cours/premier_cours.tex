%presentation
\documentclass{beamer}

%impressions
%\documentclass[handout]{beamer}
%\usepackage{pgfpages}
%\pgfpagesuselayout{2 on 1}[a4paper,border shrink=5mm]
%\setbeameroption{notes on second screen}
%\pgfpagelayout{2 on 1}[a4paper, border, shrink=5mm]
% vue sur http://wwwtaketorg/spip/articlephp3?id_article=30...
%\usepackage[T1]{fontenc}
\usepackage[frenchb]{babel}
\usepackage[utf8x]{inputenc} % Pour pouvoir taper les accents sans faire de combinaison
%\usepackage{arev}
%\usepackage{aeguill}
%mode code
\usepackage{listings}

%mode verbatim
\usepackage{moreverb}

%\usepackage[darktab]{beamerthemesidebar}
%\leftsidebar
%\usetheme{Hannover}
%\usetheme{Warsaw}
%\usetheme{PaloAlto}
\usetheme{JuanLesPins}
%\usetheme{Antibes}
%\usetheme{Shingara}
%\usetheme{Berlin}
%\usetheme{Oxygen}
\usepackage{thumbpdf}
\usepackage{wasysym}
\usepackage{ucs}
\usepackage{pgfarrows,pgfnodes,pgfautomata,pgfheaps,pgfshade}
\usepackage{verbatim}
\usepackage{color}

\title{Les Bases du Ruby}
\author{Cyril Mougel}

\lstset{
  breaklines=true
    , language=ruby
    , numbers=left
    , tabsize=2
    , basicstyle=\small\ttfamily
    , keywordstyle=\color{blue}
    , commentstyle=\color{green}
    , stringstyle=\color{red}
    , identifierstyle=\ttfamily
    , columns=fixed
    , showstringspaces=false
}

\begin{document}

\begin{frame}
  \titlepage
\end{frame}

\Large{}

\begin{frame}
	\frametitle{L'historique}
	\begin{itemize}
		\item Créé en 1993 par Yukihiro Matsumoto dit \og{}Matz\fg{}
		\item Langage de scripting de haut niveau
		\item Plus qu'orienté objet: tout est objet
        \item Applique le principe de moindre surprise (POLS, \emph{principle of
                least surprise})
        \item Fonctionne sur les plateformes les plus populaires du marché (Linux, Windows,
                Mac)
	\end{itemize}
\end{frame}

\begin{frame}
  \frametitle{G\'en\'eralit\'e sur le langage}
  \begin{itemize}
    \item typage dynamique
    \item pas d'obligation de parenthèse
    \item pas de main
  \end{itemize}
\end{frame}

\begin{frame}
  \begin{beamerboxesrounded}{Premier programme}
    \lstinputlisting[numbers=none,basicstyle=\tiny]{first_prog.rb}
  \end{beamerboxesrounded}
\end{frame}

\begin{frame}
  \frametitle{irb}
  \begin{itemize}
    \item console intéractive
    \item configurable
    \item .irbc
  \end{itemize}
\end{frame}

\begin{frame}
  \begin{beamerboxesrounded}{.irbc}
    \lstinputlisting[numbers=none,basicstyle=\tiny]{irbc.rb}
  \end{beamerboxesrounded}
\end{frame}

\section{Les Nombres}

\begin{frame}
  \begin{itemize}
    \item Integer
    \item Float
    \item Op\'erateur ( * + - / \% )
  \end{itemize}
\end{frame}

\begin{frame}
  \begin{beamerboxesrounded}{Les nombres}
    \lstinputlisting[numbers=none,basicstyle=\tiny]{number.rb}
  \end{beamerboxesrounded}
\end{frame}

\section{Les opérateurs}

\begin{frame}
  \begin{itemize}
    \item and or \&\& \textbar\textbar
    \item \textless= \textless \textgreater \textgreater=
    \item \textless=\textgreater === ==
  \end{itemize}
\end{frame}

\begin{frame}
  \begin{beamerboxesrounded}{Les op\'erateurs}
    \lstinputlisting[numbers=none,basicstyle=\tiny]{operator.rb}
  \end{beamerboxesrounded}
\end{frame}

\begin{frame}
  \begin{itemize}
    \item combinaison des op\'erateurs
    \item pas de ++ ou de - -
  \end{itemize}
\end{frame}

\begin{frame}
  \begin{beamerboxesrounded}{Combinaison d'op\'erateur}
    \lstinputlisting[numbers=none,basicstyle=\tiny]{combinate_operator.rb}
  \end{beamerboxesrounded}
\end{frame}

\section{Les Strings}

\begin{frame}
  \begin{itemize}
    \item Diff\'erence entre " et '
    \item Concat\'enation
    \item Multiplication
  \end{itemize}
\end{frame}

\begin{frame}
  \begin{beamerboxesrounded}{Les strings}
    \lstinputlisting[numbers=none,basicstyle=\tiny]{string.rb}
  \end{beamerboxesrounded}
\end{frame}

\section{Les Symbols}

\begin{frame}
  \begin{itemize}
    \item Identifiant unique
    \item Presque comme des strings
    \item Utilis\'e  pour les Hash
  \end{itemize}
\end{frame}

\begin{frame}
  \begin{beamerboxesrounded}{Les symbols}
    \lstinputlisting[numbers=none,basicstyle=\tiny]{symbols.rb}
  \end{beamerboxesrounded}
\end{frame}


\section{L'assignation}

\begin{frame}
  \begin{itemize}
    \item De gauche à droite
    \item multiple
  \end{itemize}
\end{frame}

\begin{frame}
  \begin{beamerboxesrounded}{L'assignation}
    \lstinputlisting[numbers=none,basicstyle=\tiny]{assign.rb}
  \end{beamerboxesrounded}
\end{frame}

\section{Récupération de l'entrée standard}

\begin{frame}
  \begin{itemize}
    \item STDOUT
    \item gets
    \item chomp
  \end{itemize}
\end{frame}

\begin{frame}
  \begin{beamerboxesrounded}{R\'ecup\'eration de l'entr\'ee standard}
    \lstinputlisting[numbers=none,basicstyle=\tiny]{input.rb}
  \end{beamerboxesrounded}
\end{frame}

\section{Les variables}

\begin{frame}
  \begin{itemize}
    \item Variable locale
    \item Variable globale
    \item Variable d'instance
    \item Variable de classe
    \item Constantes
  \end{itemize}
\end{frame}

\begin{frame}
  \begin{beamerboxesrounded}{Les variables}
    \lstinputlisting[numbers=none,basicstyle=\tiny]{variable.rb}
  \end{beamerboxesrounded}
\end{frame}

\begin{frame}
  \begin{beamerboxesrounded}{Les variables d'instance}
    \lstinputlisting[numbers=none,basicstyle=\tiny]{variable_2.rb}
  \end{beamerboxesrounded}
\end{frame}


\section{Les methodes}

\begin{frame}
  \frametitle{Les m\'ethodes}
  \begin{itemize}
    \item Avec ou sans arguments
    \item Avec des arguments optionnels
  \end{itemize}
\end{frame}

\begin{frame}
  \begin{beamerboxesrounded}{Les m\'ethodes}
    \lstinputlisting[numbers=none,basicstyle=\tiny]{method.rb}
  \end{beamerboxesrounded}
\end{frame}

\section{Les classes}

\begin{frame}
  \frametitle{Les Classes}
  \begin{itemize}
    \item Tout est classes, même nil
    \item H\'eritage
  \end{itemize}
\end{frame}

\begin{frame}
  \begin{beamerboxesrounded}{Les classes}
    \lstinputlisting[numbers=none,basicstyle=\tiny]{class.rb}
  \end{beamerboxesrounded}
\end{frame}

\begin{frame}
  \frametitle{Ouverture de classes}
  \begin{itemize}
    \item Les classes sont ouvertes
    \item Ajout de m\'ethode a la vol\'ee sur une instance
    \item Ajout de m\'ethode a la vol\'ee sur la classe
  \end{itemize}
\end{frame}

\begin{frame}
  \begin{beamerboxesrounded}{Ouverture de classe}
    \lstinputlisting[numbers=none,basicstyle=\tiny]{open_class.rb}
  \end{beamerboxesrounded}
\end{frame}

\begin{frame}
  \begin{beamerboxesrounded}{Ouverture de classe}
    \lstinputlisting[numbers=none,basicstyle=\tiny]{open_instance.rb}
  \end{beamerboxesrounded}
\end{frame}

\begin{frame}
  \frametitle{Self}
  \begin{itemize}
    \item l'instance de l'objet
  \end{itemize}
\end{frame}

\begin{frame}
  \begin{beamerboxesrounded}{Self}
    \lstinputlisting[numbers=none,basicstyle=\tiny]{self.rb}
  \end{beamerboxesrounded}
\end{frame}


\end{document}
