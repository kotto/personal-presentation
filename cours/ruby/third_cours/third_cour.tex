%presentation
\documentclass{beamer}

%impressions
%\documentclass[handout]{beamer}
%\usepackage{pgfpages}
%\pgfpagesuselayout{2 on 1}[a4paper,border shrink=5mm]
%\setbeameroption{notes on second screen}
%\pgfpagelayout{2 on 1}[a4paper, border, shrink=5mm]
% vue sur http://wwwtaketorg/spip/articlephp3?id_article=30...
%\usepackage[T1]{fontenc}
\usepackage[frenchb]{babel}
\usepackage[utf8x]{inputenc} % Pour pouvoir taper les accents sans faire de combinaison
%\usepackage{arev}
%\usepackage{aeguill}
%mode code
\usepackage{listings}

%mode verbatim
\usepackage{moreverb}

%\usepackage[darktab]{beamerthemesidebar}
%\leftsidebar
%\usetheme{Hannover}
%\usetheme{Warsaw}
%\usetheme{PaloAlto}
\usetheme{JuanLesPins}
%\usetheme{Antibes}
%\usetheme{Shingara}
%\usetheme{Berlin}
%\usetheme{Oxygen}
\usepackage{thumbpdf}
\usepackage{wasysym}
\usepackage{ucs}
\usepackage{pgfarrows,pgfnodes,pgfautomata,pgfheaps,pgfshade}
\usepackage{verbatim}
\usepackage{color}

\title{D\'ebuter en Ruby}
\author{Cyril Mougel}

\setbeamertemplate{blocks}[rounded]%
[shadow=false]


\lstset{
  breaklines=true
    , language=ruby
    , numbers=left
    , tabsize=2
    , basicstyle=\small\ttfamily
    , keywordstyle=\color{blue}
    , commentstyle=\color{green}
    , stringstyle=\color{red}
    , identifierstyle=\ttfamily
    , columns=fixed
    , showstringspaces=false
}

\begin{document}

\begin{frame}
  \titlepage
\end{frame}


\section{Array}

\begin{frame}
  \frametitle{Array}
  \begin{block}{Liste ordonn\'e}
    Array est un object qui correspond à une liste ordonn\'e d'\'element
  \end{block}
  \begin{block}{Diff\'erente instances possible}
    La liste ne contient pas obligatoirement la même classe. Il peux y avoir des instances de classes diff\'erente.
  \end{block}
  \begin{block}{Accesseur possible par l'index}
    Rien n'oblige à parcourir la liste dans l'ordre. On peux acc\'eder à la liste par son index ou la parcourir dans les deux sens.
  \end{block}
  \begin{block}{index commence à 0}
    L'index d'une liste commence à 0
  \end{block}
\end{frame}

\begin{frame}
  \begin{beamerboxesrounded}{Array}
    \lstinputlisting[numbers=none,basicstyle=\tiny]{array.rb}
  \end{beamerboxesrounded}
\end{frame}

\begin{frame}
  \frametitle{Quelque m\'ethode d'Array}
  \begin{block}{Array#each}
    La m\'ethode Array#each permet d'it\'erer sur tous les items de la liste. Elle est exactement identique à la m\'ethode \verb?for?. La syntaxe est par contre diff\'erente et privil\'egi\'e dans le monde Ruby.
  \end{block}
\end{frame}

\section{ARGV}


\begin{frame}
  \frametitle{Ouverture de classe}
  Permettre la possibilit\'e de cr\'eer des voitures directement à partir d'un entier comment indiquer dans le test suivant.
\end{frame}
\begin{frame}
  \begin{beamerboxesrounded}{Super}
    \lstinputlisting[numbers=none,basicstyle=\tiny]{test_car_11.rb}
  \end{beamerboxesrounded}
\end{frame}

\section{alias}
\section{argument args*}

\end{document}
